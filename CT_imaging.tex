\documentclass[a4paper,12pt]{article}
\usepackage[utf8]{inputenc}
\usepackage{enumitem}
\usepackage{amsmath}
\usepackage{graphicx}

\title{CT Imaging}
\author{}
\date{}

\begin{document}

\maketitle

\section*{Hounsfield Units (HU)}
\begin{itemize}[leftmargin=1.5em]
    \item are used to segment CT data by dividing the image into regions with similar X-ray density.
    \item Each voxel in a CT scan is assigned a value based on its X-ray attenuation, measured in HU.
\end{itemize}

\section*{Thresholding ( = Threshold Segmentation)}
\begin{itemize}[leftmargin=1.5em]
    \item Discretization of HU values to classify different tissue types based on intensity.
\end{itemize}

\section*{Window / Level Settings}
\begin{itemize}[leftmargin=1.5em]
    \item \textbf{Window:} Defines the range of Hounsfield Unit values mapped to the grayscale, enhancing contrast for specific tissues.
    \item \textbf{Level (aka Window Center):} Determines the midpoint of the window range, setting the overall brightness of the image.
    \item In \textbf{3D Slicer}:
    \begin{itemize}
        \item Click the colorful icon in the toolbar.
        \item Mouse movement:
        \begin{itemize}
            \item Up/Down: Adjust window (contrast)
            \item Left/Right: Adjust level (brightness)
        \end{itemize}
    \end{itemize}
\end{itemize}

\section*{Registration}
\begin{itemize}[leftmargin=1.5em]
    \item has already been performed on the dataset derivatives.
    \item aligns multiple images spatially to enable correct comparison or fusion.
\end{itemize}

\section*{Image Overlay in 3D Slicer}
\begin{itemize}[leftmargin=1.5em]
    \item Open the View Controllers panel
    \item Expand a channel
    \item Click double-ring icon
    \item Select the images to overlay
\end{itemize}

\end{document}
